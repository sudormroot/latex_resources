\documentclass{article}

\usepackage{pgf}

%\usepackage[T1]{fontenc}

%\usepackage[english]{babel}
\usepackage{fp,xstring}
\usepackage[autolanguage]{numprint}

\begin{document}

\newcommand{\varA}{87}
\newcommand{\varB}{17}

\noindent
Simple demon use pgf package \\
\varA \ divided by \varB \ is approximately \pgfmathparse{int(round(\varA / \varB))}\pgfmathresult.
\noindent
the result is \pgfmathresult

\newcommand{\varC}{\pgfmathresult}
\noindent
the varC is \varC

\newcommand{\varD}{99}
\newcommand{\varE}{7}
\noindent
\varD \ divided by \varE \ is approximately \pgfmathparse{int(round(\varD / \varE))}\pgfmathresult.
\noindent
the varC is \varC


----------
\noindent
Simple demo using fp package

\def\mynumber{120,000.00}
\def\mydiv{17}
$\mynumber/\mydiv= %
\StrDel\mynumber,[\mynumber] %
\FPeval\myresult{round(\mynumber/\mydiv,0)} %
\numprint\myresult$

\end{document}