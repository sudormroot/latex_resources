\documentclass[border=10pt]{standalone}
%%%<
\usepackage{verbatim}
%%%>
\usepackage{pgfplots}
\pgfplotsset{compat=1.8}
\usepackage{amsmath}
\begin{comment}
:Title: Contour on a surface
:Tags: 2D;PGFPlotstable;Styles
:Author: Jake
:Slug: contour-surface

We will combine surface and contour plots. We would like the contour lines
drawn on the surface to be hidden by the surface in the foreground but not
in the background, like looking into such a "valley".

It is very difficult with the current version of pgfplots. The simple reason
is that z-buffering is not fully implemented.

We can do our own z-buffering (which can be quite cumbersome). This means
that you have to draw the parts in terms of their appearance on the screen,
and thus a lot of double drawings.

To run this example for the first time you have to do the following:

- GNUPLOT must be installed on your system. Try typing ``gnuplot`` on the
  command line to see if it's installed. Windows users may have to
  rename ``wgnuplot`` to ``gnuplot``.
- You must allow TeX to run external programs. The command line option to
  enable this is usually ``--shell-escape`` or ``--enable-write18``.

PGF will call GNUPLOT for you and store the data in a file. Next time you
compile the example, data will be loaded from the generated file. See
section 11.12.3 in the manual for more information.

This code was written by Jake and JJ Fleck on TeX.SE.
\end{comment}
\begin{document}
\pgfplotsset{
  compat=1.8,
  colormap={whitered}{color(0cm)=(white); color(1cm)=(orange!75!red)}
}

\begin{tikzpicture}
  \begin{axis}[
    colormap name=whitered,
    3d box,
    width=15cm,
    view={25}{25},
    enlargelimits=false,
    grid=major,
    domain=-0.5:4.7,
    y domain=-2:2,
    samples=21,
    xlabel=$x$,
    ylabel=$\dot{x}$,
    zlabel={$\text{E}_{\text{m}}$},
    colorbar,
    colorbar style={
        at={(1,0)},
        anchor=south west,
        height=0.1*\pgfkeysvalueof{/pgfplots/parent axis height},
        title={$\text{E}_{\text{m}}(x,\dot{x})$}
        }
    ]
    \addplot3 [y domain = 0:2, surf]
      {-0.7+4*exp(-0.5*(x+3))*(3*cos(4*x*180/pi)+2.5*cos(2*x*180/pi))
       + 0.5*y*y*4};

    \addplot3 [y domain = 0:2, contour gnuplot = {number=14, labels={false},
      draw color = black}, samples = 21, ]
      {-0.7+4*exp(-0.5*(x+3))*(3*cos(4*x*180/pi)+2.5*cos(2*x*180/pi))
       + 0.5*y*y*4};

    \addplot3 [domain = -0.5:4.7, samples = 31, samples y = 0, thick, smooth]
        (x,-2,{-0.6+4*exp(-0.5*(x+3))*(3*cos(4*x*180/pi)+2.5*cos(2*x*180/pi))});
    \addplot3 [contour gnuplot = {number=14, labels={false}, draw color=black},
        samples=21,z filter/.code={\def\pgfmathresult{20}}]
        {-0.7+4*exp(-0.5*(x+3))*(3*cos(4*x*180/pi)+2.5*cos(2*x*180/pi))
         + 0.5*y*y*4};
    \addplot3 [y domain=-2:0,surf]
        {-0.7+4*exp(-0.5*(x+3))*(3*cos(4*x*180/pi)+2.5*cos(2*x*180/pi))
         + 0.5*y*y*4};
    \addplot3 [domain = 0:.25, contour gnuplot = {number=14,labels={false},
        draw color=black}, samples=21]
        {-0.7+4*exp(-0.5*(x+3))*(3*cos(4*x*180/pi)+2.5*cos(2*x*180/pi))
         + 0.5*y*y*4};
  \end{axis}
\end{tikzpicture}
\end{document}
