\documentclass[border=10pt]{standalone}
%%%<
\usepackage{verbatim}
\usepackage{filecontents}
\begin{filecontents}{data.dat}
3.045784        3.415896
3.405784        4.025693
3.785784        4.522530
4.125784        5.538449
4.485784        6.704992
4.805784        6.978939
5.145784        7.113496
5.425784        8.916397
6.065784        9.487712
6.365784        10.876397
6.685784        10.693497
7.025784        11.364131
7.345784        11.442530
7.665784        12.582530
8.005784        13.125693
8.225784        13.738450
8.585784        14.247891
8.865784        14.982530
\end{filecontents}
%%%>
\usepackage{pgfplots}
\pgfplotsset{width=7cm,compat=1.6}
\begin{comment}
:Title: Mathematica 9 style plots
:Tags: 2D;ycomb
:Author: Jake
:Slug: mathematica-9-style

Here's a way of doing Mathematica 9 style plots with PGFPlots: the first plot
is simply a ycomb plot, and the second we can draw using separate plots for the
areas (with a trailing \closedcycle to get the filling right) and the lines.

This code was written by Jake on TeX.SE.
\end{comment}

\begin{document}
\begin{tikzpicture}
  \begin{axis}[
    ymin=0,
    xtick=\empty,
    ytick=\empty,
    axis background/.style={fill=gray!10},
  ]
  \addplot [ycomb, red, very thick, mark=*, mark options={red!60!black}] table {data.dat};
  \end{axis}
\end{tikzpicture}

\begin{tikzpicture}
  \begin{axis}[
    ymin=0,
    xtick=\empty,
    ytick=\empty,
    axis background/.style={fill=gray!10},
    enlarge x limits=false
  ]
  \addplot [draw=none, fill=orange!40!yellow] table {data.dat} \closedcycle;
  \addplot [draw=gray, very thick] table {data.dat};

  \addplot [draw=none, fill=red!60] table [
    x expr=\thisrowno{0}+8.865784-3.045784,
    y expr=\thisrowno{1}+14.982530-3.415896
  ] {data.dat} \closedcycle;
  \addplot [draw=red!70!black, very thick] table [
    x expr=\thisrowno{0}+8.865784-3.045784,
    y expr=\thisrowno{1}+14.982530-3.415896
  ] {data.dat};
    \draw node[append after command={ (a) -| (axis cs:10.5,13.5)}, font=\sffamily]
    (a) at (axis cs:6,16) {Forecast};
\end{axis}
\end{tikzpicture}
\end{document}
